\note{learned adaptation is extinguished when no longer necessary}
\note{old memories may no longer be useful, and in some cases may be maladaptive}
\note{Classically conditioned eyeblink responses undergo extinction after prolonged exposure to the conditioned stimulus in the absence of the unconditioned stimulus}
\note{After learning, head movements in the absence of visual stimulation caused a loss of the learned eye movement response. When the learned gain was low, this reversal of learning occurred only when head movements were delivered, and not when the head was held stationary in the absence of visual input, suggesting that this reversal is mediated by an active, extinction-like process}
\note{It is often adaptive to retain memories over a long period of time. When environmental circumstances change, however, old memories may no longer be useful, and in some cases may be maladaptive. Therefore, an ideal learning system should have a mechanism for suppressing old memories. Old memories can be abolished or suppressed through passive forgetting or through an active process such as extinction, the reduction of a conditioned response that occurs when the learned association between a cue and reinforcement is degraded.}
\note{we examined whether learned changes in the amplitude, or gain, of the vestibulo-ocular reflex (VOR), a well-studied cerebellum-dependent motor learning paradigm, exhibit a process of reversal analogous to extinction.}


\item VOR adaptation has a non-perfect performance, with a residual error proportional to the amount of cerebellar action required
